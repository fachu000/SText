  
%%%%%%%%%%%%%% WARNING %%%%%%%%%%%%%%%%%%%%%%%%%%%%%%%%%%%%%%%%%%%%%%%%%%%

% Before compling, comment the line
% \usepackage{hyperref}
% in draft.tex. There is a conflict between the xcite and hyperref packages. 

%%%%%%%%%%%%%% SETTINGS %%%%%%%%%%%%%%%%%%%%%%%%%%%%%%%%%%%%%%%%%%%%%%%%%%

% SET FOLLOWING VARIABLE TO 1 FOR EDIT MODE AND 0 FOR VIEW MODE
% (erase .bbl and .aux files in this folder every time
% you switch from one mode to another, otherwise you get
% an error)
% 
% One of the features of the edit mode is that it shows citations in
% boldface if they are not defined in the bibtex file, like the [?] in
% view mode. Different from the view mode, the edit mode indicates
% which is the missing reference.
% 
% Note that the cross citations will only work correctly if the paper
% (rather than this letter) has been compiled with edit mode = 0.
\def\editmode{0}
\def\singlenarrowcol{0}

% If 0, intermediate steps are not shown. 
% If 1, intermediate steps are shown in purple
% Note that one cannot have "\\" inside an "intermed" environment. Just put `\jumpline` before or after the environment instead of "\\".  The same for "&" and "\alignchar"
\def\intermediatesteps{0}

% set the following variable to the filenames, separated by commas, of
% the bibfiles that contain your references.  
\def\bibfilenames{../paper/bibman_refs}

% ========================================================================
%                          TEMPLATE
% ========================================================================
\if\singlenarrowcol0
\documentclass[11pt]{article}
\addtolength{\textwidth}{5.2cm} \addtolength{\textheight}{60pt}
\addtolength{\hoffset}{-2.8cm} \addtolength{\voffset}{-30pt}
\else
\documentclass[onecolumn]{IEEEtran}
    \usepackage[pass]{geometry}
    \newgeometry{textwidth=252pt, textheight=672pt}
\fi
% ========================================================================

%% Check the following document for templates such as custom
%% enumerations or how to include figures or tables.
\usepackage{../paper/include_v8}
\usepackage{../paper/notation}

%%  
\renewcommand{\theequation}{R\arabic{equation}} 

% ========================================================================
%                          (CROSS)-CITATIONS
% ========================================================================
% At this point, you need to set \hyperlinks and \editmode to 0 in the paper. As
% per the documentation of xr-hyper, it will be possible to do it  in a
% different way in the future. 
%
%To cite a reference from the paper use \cite{P-<key>}. It will be % displayed
%e.g. as [4]. To create a new bibliographic entry in this % letter, write
%\cite{<key>}. It will be displayed e.g. as [R4]. 
\usepackage{xr,xcite}
\externaldocument[P-]{../paper/paper}
\externalcitedocument[P-]{../paper/paper}

\makeatletter
\renewcommand\@bibitem[1]{\item\if@filesw \immediate\write\@auxout
    {\string\bibcite{#1}{R\the\value{\@listctr}}}\fi\ignorespaces}
\def\@biblabel#1{[R#1]}
\makeatother

% ========================================================================

\begin{document}


\begin{center}
    \Large \textbf{T-SP-22369-2017: Authors' Response to Reviewers'~Comments}\\
    \today
\end{center}
\acom{Please make sure to use the right code in the title and that the comments are properly numbered.}

We would like to thank the reviewers and the associate editor for
their constructive feedback on the original submission, which
contributed considerably to improving the manuscript. Changes are
marked in {\color{blue}blue} in the revised manuscript. They mainly
comprise some clarifications and new simulations.

%% \begin{itemize}
%% \item[\textbf{c1)}] 
%% \end{itemize}

\noindent Unless stated otherwise, the equations, figures, remarks,
propositions and (sub-)sections, are referenced with respect to the
revised paper. Notation \cite{P-kay1}
and \eqref{P-eq:testequation} refers to bibliographic entries and equations in
the manuscript respectively, whereas \cite{kay2} and \eqref{eq:sup}
refer to the present document.

%%%%%%%%%%%%%%%%%%%%%%%%%%%   REVIEWER 1   %%%%%%%%%%%%%%%%%%%%%%%%%%%
\section*{Responses to Reviewer 1's Comments}
\begin{rcomment}
    General comment
\end{rcomment}

\vspace*{0.5em} \noindent
\begin{response}
    We thank the reviewer for his/her positive assessment and feedback.

    Test equation:
    \begin{align}
        \label{eq:sup}
        \sup[0,1) = 1
    \end{align}

\end{response}

\begin{enumerate}[label=R1.\arabic*] %[(1)]%[label=R1.\alph]%[{R1}.1]

    \item \begin{rcomment}
              Comment 1.
          \end{rcomment}

          \begin{response}
              \begin{bullets}
                  \blt[cite from the paper]Test citations from the paper:~\cite{P-kay1}.
                  \blt[new citation]Test citations that create new bibliographic entries in this letter:~\cite{wilson2009regularization,wilson2010tomography}.

              \end{bullets}

          \end{response}

    \item \begin{rcomment}
              Comment 2.
          \end{rcomment}
          \begin{response}
              Response here.
          \end{response}
\end{enumerate}

%%%%%%%%%%%%%%%%%%%%%%%%%%%   REVIEWER 2   %%%%%%%%%%%%%%%%%%%%%%%%%%%
\section*{Responses to Reviewer 2's Comments}

\begin{rcomment}
    General comment
\end{rcomment}

\vspace*{0.5em} \noindent
\begin{response}
    We thank the reviewer for his/her positive assessment and feedback.
\end{response}

\begin{enumerate}[label=R2.\arabic*]

    \item \begin{rcomment}
              Comment 1.
          \end{rcomment}

          \begin{response}
              Response here.
          \end{response}

    \item \begin{rcomment}
              Comment 2.
          \end{rcomment}

          \begin{response}
              Response here.
          \end{response}
\end{enumerate}

\printmybibliography
\end{document}
