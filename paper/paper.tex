
%%%%%%%%%%%%%% SETTINGS %%%%%%%%%%%%%%%%%%%%%%%%%%%%%%%%%%%%%%%%%%%%%%%%%%

% SET FOLLOWING VARIABLE TO 1 FOR EDIT MODE AND 0 FOR VIEW MODE
% (erase .bbl and .aux files in this folder every time
% you switch from one mode to another, otherwise you get
% an error)
% One of the features of the edit mode is that it shows citations in
% boldface if they are not defined in the bibtex file, like the [?] in
% view mode. Different from the view mode, the edit mode indicates
% which is the missing reference.
%
% Also, when writing the response letter, you may need to set this
% variable to 0 in order for cross-referencing to work.
\def\editmode{0}

% set the following variable to the filenames, separated by commas, of
% the bibfiles that contain your references.  
\def\bibfilenames{bibman_refs}

% Use the following lines if you want to modify how citations are displayed
%% \let\oldcite\cite
%% \renewcommand{\cite}[1]{\textbf{\oldcite{#1}}}

% ========================================================================
%                          CHOOSE THE TEMPLATE
% ========================================================================
% Wether to use the format of the IEEE SPS
\def\spsformat{0}
\def\singlenarrowcol{1} % only applicable if spsformat == 0

\if\spsformat1

\documentclass{article}
\usepackage{spconf}

\else


\if\singlenarrowcol0
%\documentclass[journal]{IEEEtran}
%\documentclass[draftclsnofoot,onecolumn,12pt]{IEEEtran}
\documentclass[11pt,final,onecolumn]{IEEEtran}
%\documentclass[10pt,final,twocolumn]{IEEEtran}
%\documentclass[twocolumn,twoside]{IEEEtran}
%% \documentclass[11pt]{article}
%% \addtolength{\textwidth}{5.2cm} \addtolength{\textheight}{60pt}
%% \addtolength{\hoffset}{-2.8cm} \addtolength{\voffset}{-30pt}

% The following produces a one-column document whose column size is roughly the
% column size in the double-column document. Thus, the number of pages is
% roughly twice the number of pages in the double-column document. 
% \documentclass[onecolumn]{IEEEtran}
%  \usepackage[pass]{geometry}
%  \newgeometry{textwidth=252pt, textheight=672pt}


\IEEEoverridecommandlockouts
% The preceding line is only needed to identify funding in the first footnote. If that is unneeded, please comment it out.
\else

\documentclass[onecolumn]{IEEEtran}
    \usepackage[pass]{geometry}
    \newgeometry{textwidth=252pt, textheight=672pt}

\fi
\fi
% ========================================================================

\usepackage{include_v7}

% Comment the following line if you are going to use xcite
%\usepackage{hyperref}

\begin{document}

%%%%%%%%%%%%%%%%%%%%%%%%%%%%%%%%%%%%%%%%%%%%%%%%%%%%%%%%%%%%%%%%%%%%%
\if\spsformat1
    \title{Title}
    \name{Author(s) Name(s)\thanks{Thanks to XYZ agency for funding.}}
    \address{Author Affiliation(s)}
\else
    \title{Title
        \thanks{Identify applicable funding agency here. If none, delete this.}}

    % Use the following for conferences:
    % \author{\IEEEauthorblockN{1\textsuperscript{st} Given Name Surname}
    % \IEEEauthorblockA{\textit{dept. name of organization (of Aff.)} \\
    % \textit{name of organization (of Aff.)}\\
    % City, Country \\
    % email address or ORCID}
    % \and
    % \IEEEauthorblockN{2\textsuperscript{nd} Given Name Surname}
    % \IEEEauthorblockA{\textit{dept. name of organization (of Aff.)} \\
    % \textit{name of organization (of Aff.)}\\
    % City, Country \\
    % email address or ORCID}
    % }

    % Use the following for journal articles:
    \author{Author(s) Name(s)\thanks{Thanks to XYZ agency for funding.}}
\fi

\maketitle
%%%%%%%%%%%%%%%%%%%%%%%%%%%%%%%%%%%%%%%%%%%%%%%%%%%%%%%%%%%%%%%%%%%%%


\begin{abstract}
    Toggle the variable \texttt{editmode} in the main source file to
    show/hide the bullets and labels. If switching the edit mode results
    in a compilation error, just delete the .aux file.
\end{abstract}

\if\spsformat0
\begin{IEEEkeywords}
    One, two, three, four, five
\end{IEEEkeywords}
\fi

\section{Introduction}
\begin{bullets}
    \blt[overview]
    \blt[motivation]
    \blt[literature] [test citation~\cite{kay1}]
    \blt[contributions]
    \blt[paper structure]
    \blt[notation]
\end{bullets}

\section{Model and Problem Formulation}

\cmt{model}Test equation:
\begin{align}
    \label{eq:euler}
    e^{j\pi}=-1
\end{align}

\cmt{problem formulation}

\section{Proposed Solution}

\section{Numerical Experiments}

\cmt{simulation setup}
\begin{bullets}
    \blt[data generation]
    \blt[tested algorithms]
    \blt[performance metrics]
\end{bullets}

\cmt{description of the experiments}

\section{Conclusions}


\printmybibliography
\end{document}



