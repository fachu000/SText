
%%%%%%%%%%%%%% SETTINGS %%%%%%%%%%%%%%%%%%%%%%%%%%%%%%%%%%%%%%%%%%%%%%%%%%

% SET FOLLOWING VARIABLE TO 1 FOR EDIT MODE AND 0 FOR VIEW MODE
% (erase .bbl and .aux files in this folder every time
% you switch from one mode to another, otherwise you get
% an error)  
% One of the features of the edit mode is that it shows citations in
% boldface if they are not defined in the bibtex file, like the [?] in
% view mode. Different from the view mode, the edit mode indicates
% which is the missing reference.
%
% Also, when writing the response letter, you may need to set this
% variable to 0 in order for cross-referencing to work.
\def\editmode{1}

% The following variable needs to be set to 0 to compile the response letter.
% Else, cross-referencing will not work (xcite is incompatible with href).
\def\hyperlinks{1}  

% Wether to use the format of the IEEE SPS:
\def\spsformat{0}

% Wether to use a single column with the same width as the columns in the
% double-column format. This is useful for editing and reviewing the
% document in a laptop screen. 
\def\singlenarrowcol{1} % only applicable if spsformat == 0

% set the following variable to the filenames, separated by commas, of
% the bibfiles that contain your references.  
\def\bibfilenames{bibman_refs}

% If \journal == 0, only the conference content is shown. 
% If \journal == 1, only the journal content is shown.
% If \journal == 2, both the conference and journal content are shown. The journal content is shown in dark yellow (as old newspapers), the conference content in blue/green (as the sea). 
\def\journal{2}

% If 0, intermediate steps are not shown. 
% If 1, intermediate steps are shown in purple
% Note that one cannot have "\\" inside an "intermed" environment. Just put `\jumpline` before or after the environment instead of "\\".  The same for "&" and "\alignchar"
\def\intermediatesteps{1}

% Consider using the following package to equalize the length of the columns of
% the last page of the document. 
% \usepackage{flushend}

% Use the following lines if you want to modify how citations are displayed
%% \let\oldcite\cite
%% \renewcommand{\cite}[1]{\textbf{\oldcite{#1}}}

% ========================================================================
%                          CHOOSE THE TEMPLATE
% ========================================================================

\if\spsformat1

\documentclass{article}
\usepackage{spconf}

\else


\if\singlenarrowcol0
\documentclass[journal]{IEEEtran}
%\documentclass[draftclsnofoot,onecolumn,12pt]{IEEEtran}
%\documentclass[11pt,final,onecolumn]{IEEEtran}
%\documentclass[10pt,final,twocolumn]{IEEEtran}
%\documentclass[twocolumn,twoside]{IEEEtran}
%% \documentclass[11pt]{article}
%% \addtolength{\textwidth}{5.2cm} \addtolength{\textheight}{60pt}
%% \addtolength{\hoffset}{-2.8cm} \addtolength{\voffset}{-30pt}

% The following produces a one-column document whose column size is roughly the
% column size in the double-column document. Thus, the number of pages is
% roughly twice the number of pages in the double-column document. 
% \documentclass[onecolumn]{IEEEtran}
%  \usepackage[pass]{geometry}
%  \newgeometry{textwidth=252pt, textheight=672pt}


\IEEEoverridecommandlockouts
% The preceding line is only needed to identify funding in the first footnote. If that is unneeded, please comment it out.
\else

\documentclass[onecolumn]{IEEEtran}
    \usepackage[pass]{geometry}
    \newgeometry{textwidth=252pt, textheight=672pt}

\fi
\fi
% ========================================================================

\usepackage{include_v8}
\usepackage{notation}


\begin{document}

%% doc_body__begin

%%%%%%%%%%%%%%%%%%%%%%%%%%%%%%%%%%%%%%%%%%%%%%%%%%%%%%%%%%%%%%%%%%%%%
\if\spsformat1
    \title{Title}
    \name{Author(s) Name(s)\thanks{Thanks to XYZ agency for funding.}}
    \address{Author Affiliation(s)}
\else
    \title{Title
        \thanks{Identify applicable funding agency here. If none, delete this.}}

    % Use the following for conferences:
    % \author{\IEEEauthorblockN{1\textsuperscript{st} Given Name Surname}
    % \IEEEauthorblockA{\textit{dept. name of organization (of Aff.)} \\
    % \textit{name of organization (of Aff.)}\\
    % City, Country \\
    % email address or ORCID}
    % \and
    % \IEEEauthorblockN{2\textsuperscript{nd} Given Name Surname}
    % \IEEEauthorblockA{\textit{dept. name of organization (of Aff.)} \\
    % \textit{name of organization (of Aff.)}\\
    % City, Country \\
    % email address or ORCID}
    % }

    % Use the following for journal articles:
    \author{Author(s) Name(s)\thanks{Thanks to XYZ agency for funding.}}
\fi

\maketitle
%%%%%%%%%%%%%%%%%%%%%%%%%%%%%%%%%%%%%%%%%%%%%%%%%%%%%%%%%%%%%%%%%%%%%




\begin{abstract}
    Toggle the LaTeX variable \texttt{editmode} in the main source file to
    show/hide the bullets and labels. If switching the edit mode results
    in a compilation error, just delete the .aux file.

    If the LaTeX variable \texttt{singlenarrowcol} is 1, then each page contains
    a single column whose size equals the size of each of the columns in a
    double-column layout. This is convenient for editing and reviewing the
    document in a laptop screen. The number of pages when
    \texttt{singlenarrowcol} is 1 is roughly twice the number of pages when
    \texttt{singlenarrowcol} is 0.

\end{abstract}

\if\spsformat0
    \begin{IEEEkeywords}
        One, two, three, four, five
    \end{IEEEkeywords}
\fi

\section{Introduction}
\begin{bullets}
    \blt[overview]

    \blt[motivation]

    \blt[literature] [test citation~\cite{kay1}]

    \blt[novelty]Write here the major claim of the paper. This should be as general as possible but it should be true. For example, consider the following claims:
    \begin{bullets}%
        \blt[general claim] (C1) "This is the first paper to address the problem of path planning for aerial relays", and

        \blt[specific claim] (C2)  "This is the first paper to address the problem of path planning for multiple aerial relays using reinforcement learning in urban environments with constraints on the user rate".
    \end{bullets}%
    (C1) is clearly more general since it is saying that nobody has addressed the problem of path planning for aerial relays before. In contrast, (C2) is more specific since it is implicitly saying that other people have considered very similar problems such as path planning for aerial relays using reinforcement learning in urban environments but without constraints on the user rate.


    \blt[contributions]

    \blt[paper structure]Sec.~\ref{sec:model} introduces the system model and formulates the problem\ldots

    \blt[notation]
\end{bullets}

\section{Model and Problem Formulation}
\label{sec:model}

\begin{bullets}
    \blt[model]    The following equation illustrates the usage of \texttt{salign}, \texttt{\textbackslash hc} and \texttt{\textbackslash{}newcommandoa}:
    \begin{salign}[eq:globallabel]
        \label{eq:testequation}
        \testsymbol&=1\\
        \sectestsymbol &\in \{\sectestsymbol[0],\ldots,\sectestsymbol[N-1]\}
    \end{salign}

    \blt[problem formulation]For an enumeration that should be visible when not in edit mode, use \texttt{\textbackslash cmt}:
    \begin{enumerate}
        \item \cmt{first item} This goes first.
        \item \cmt{second item} This goes second.
    \end{enumerate}


\end{bullets}

\section{Proposed Solution}
\begin{bullets}%
    \blt[intermed] Intermediate steps are shown in purple if \texttt{\textbackslash intermediatesteps} is set to 1.
    \begin{align}
        (x+1)^2 + (x-1)^2
        \alignchar
        \begin{intermed}
            = x^2 + 2x + 1 + x^2 - 2x + 1
        \end{intermed}
        \jlac= 2x^2 + 2
    \end{align}
    Use \texttt{\textbackslash jumpline} instead of \texttt{\textbackslash\textbackslash} and  \texttt{\textbackslash alignchar} instead of \texttt{\&}. A short form for \texttt{\textbackslash jumpline\textbackslash alignchar} is \texttt{\textbackslash jlac}.
\end{bullets}%


\section{Analysis}

\begin{bullets}
    \blt[overview]
    \blt[journal]  Set the variable \texttt{journal} to 0, 1, or 2 to show only the conference content, only the journal content, or both in different colors, respectively.
    \begin{journalonly}
        This is the journal-only content.
    \end{journalonly}
    \begin{conferenceonly}
        This is the conference-only content.
    \end{conferenceonly}

    \blt[References]Use \texttt{\textbackslash label\{prop:XXXX\}} to label any proposition, which includes theorems, lemmas, and corollaries. To refer to it, use \texttt{\textbackslash Cref\{prop:XXXX\}}. Replace \texttt{XXXX} with the label of the proposition.  For example, next result is~\Cref{prop:mainresult}.

    \blt[main result]

    \begin{theorem}
        \label{prop:mainresult}
        If it rains, it is cloudy.
    \end{theorem}
    \begin{IEEEproof}
        \begin{conferenceonly}
            The proof is omitted due to lack of space.
        \end{conferenceonly}
        \begin{journalonly}
            The proof is in Appendix~\ref{app:proof}.
        \end{journalonly}
    \end{IEEEproof}

    \blt[corollary] Now a consequence of \Cref{prop:mainresult}:

    \begin{corollary}
        \label{prop:corollary}
        If it rains, it is cloudy.
    \end{corollary}

    We refer to it as \Cref{prop:corollary}.



\end{bullets}


\section{Numerical Experiments}

\cmt{simulation setup}
\begin{bullets}
    \blt[data generation]
    \blt[tested algorithms]
    \blt[performance metrics]
\end{bullets}


\cmt{description of the experiments}

\section{Conclusions}

\appendices


\begin{journalonly}

    \section{Proof of \cref{prop:mainresult}}
    \label{app:proof}

    This is the proof of \cref{prop:mainresult}
\end{journalonly}


%% doc_body__end
\printmybibliography


\end{document}



