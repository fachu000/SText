

%%%%%%%%%%%%%% SETTINGS %%%%%%%%%%%%%%%%%%%%%%%%%%%%%%%%%%%%%%%%%%%%%%%%%%

% SET FOLLOWING VARIABLE TO 1 FOR EDIT MODE AND 0 FOR VIEW MODE
% (erase .bbl and .aux files in this folder every time
% you switch from one mode to another, otherwise you get
% an error)
% One of the features of the edit mode is that it shows citations in
% boldface if they are not defined in the bibtex file, like the [?] in
% view mode. Different from the view mode, the edit mode indicates
% which is the missing reference.
\def\editmode{1}

% set the following variable to the filenames, separated by commas, of
% the bibfiles that contain your references.  
\def\bibfilenames{bibman_refs}

% Use the following lines if you want to modify how citations are displayed
%% \let\oldcite\cite
%% \renewcommand{\cite}[1]{\textbf{\oldcite{#1}}}

% ========================================================================
%                          CHOOSE THE TEMPLATE
% ========================================================================
% Wether to use the format of the IEEE SPS
\def\spsformat{0}

\if\spsformat1

\documentclass{article}
\usepackage{spconf}

\else
%\documentclass[journal]{IEEEtran}
%\documentclass[draftclsnofoot,onecolumn,12pt]{IEEEtran}
\documentclass[11pt,final,onecolumn]{IEEEtran}
%\documentclass[10pt,final,twocolumn]{IEEEtran}
%\documentclass[twocolumn,twoside]{IEEEtran}
%% \documentclass[11pt]{article}
%% \addtolength{\textwidth}{5.2cm} \addtolength{\textheight}{60pt}
%% \addtolength{\hoffset}{-2.8cm} \addtolength{\voffset}{-30pt}

\fi
% ========================================================================


/Users/dani/Dropbox/signal_processing_stuff/common/latex/include_v5.tex

\begin{document}

%%%%%%%%%%%%%%%%%%%%%%%%%%%%%%%%%%%%%%%%%%%%%%%%%%%%%%%%%%%%%%%%%%%%%
\title{Title}

\if\spsformat1
\name{Author(s) Name(s)\thanks{Thanks to XYZ agency for funding.}}
\address{Author Affiliation(s)}
\else
\author{Author(s) Name(s)\thanks{Thanks to XYZ agency for funding.}}
\fi

\maketitle
%%%%%%%%%%%%%%%%%%%%%%%%%%%%%%%%%%%%%%%%%%%%%%%%%%%%%%%%%%%%%%%%%%%%%


\begin{abstract}
Toggle the variable \texttt{editmode} in the main source file to
show/hide the bullets and labels. If switching the edit mode results
in a compilation error, just delete the .aux file. 
\end{abstract}


\begin{keywords}
One, two, three, four, five
\end{keywords}

\section{Introduction}
\begin{bullets}
\blt[overview]
\blt[motivation]
\blt[literature] [test citation~\cite{kay1}]
\blt[contributions]
\blt[paper structure]
\blt[notation]
\end{bullets}

\section{Model and Problem Formulation}

\cmt{model}Test equation:
\begin{align}
\label{eq:euler}
e^{j\pi}=-1
\end{align}

\cmt{problem formulation}

\section{Proposed Solution}

\section{Numerical Experiments}

\cmt{simulation setup}
\begin{bullets}
\blt[data generation]
\blt[tested algorithms]
\blt[performance metrics]
\end{bullets}

\cmt{description of the experiments}

\section{Conclusions}


\printmybibliography
\end{document}



